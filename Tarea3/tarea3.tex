\documentclass[letter, 10pt]{article}
\usepackage[latin1]{inputenc}
\usepackage[spanish]{babel}
\usepackage{amsfonts}
\usepackage{amsmath}
\usepackage[dvips]{graphicx}
\usepackage{url}
\usepackage{algorithm}
\usepackage{algorithmic}
\usepackage[top=3cm,bottom=3cm,left=3.5cm,right=3.5cm,footskip=1.5cm,headheight=1.5cm,headsep=.5cm,textheight=3cm]{geometry}


\begin{document}
\title{Simulaci\'on \\ \begin{Large}Tarea 3: M\'etodo de Aceptaci\'on y rechazo\end{Large}}
\author{Renato Rivera Mohana.}
\date{\today}
\maketitle

\section{Verificar que el m\'etodo de aceptacion y rechazo equivale a generar valores con:}
\begin{equation}
\setcounter{equation}{1}
 Y \sim U[0, a g(x)], \;  \mbox{Y aceptamos si} \; Y \leq f(x) 
\end{equation}
\\
Generar valores con la ecuacion (1) es una versi\'on ligeramente modificada del m\'etodo de aceptaci\'on y rechazo ya que: 
\begin{center}
$ Y  \sim U(0, a g(x)) $ es lo mismo que $Y = U ag(x) $ donde $ U \sim U(0,1) $
\end{center}
\begin{center}
 Por ende,  $ Y \leq f(x)$ cambia por $ U \leq f(x)/(ag(x)) $.
\end{center}

  
\section{Verificar que en el m\'etodo de aceptaci\'on y rechazo, cada iteraci\'on se acepta con probabilidad $1/a$}
Probabilidad de aceptaci\'on:
\begin{equation}
P(U\leq {f(X)\over ag(X) | X=x)} = {f(x) \over ag(x)}
\end{equation}
Entonces,
\begin{equation}
p=\int_x {f(x)\over ag(x)} g(x)\mathrm{d}x
\end{equation}
$$ = {1\over a} \int_x f(x) \mathrm{d}x$$
$$= {1\over a}.$$

\section{Verificar que la eficiencia de aceptaci\'on y rechazo es 1/a.}
La eficiencia del m\'etodo de aceptaci\'on y rechazo  esta determinado por la probabilidad de aceptaci\'on:
\begin{equation}
p=P(U\leq f(x)/(ag(x))))=P(Y\leq f(x))=1/a.
\end{equation}



\section{El n\'umero de iteraciones del m\'etodo de aceptaci\'on y rechazo, antes de aceptar sigue una ley geom\'etrica de raz\'on 1/a}
Dado que los intentos son independientes, del n\'umero de intentos, N antes de conseguir un par exitoso ocurre la siguiente distribuci\'on geom\'etrica:
\begin{equation}
P(N=n)=p(1-p)^{n-1}, \; n=1,2,...,
\end{equation}

\section{El n\'umero esperado de iteraciones es a.}
Dada la distribuci\'on geom\'etrica en (5), esta tiene un n\'umero esperado de intentos $1/p$, dado que $p=1/a$, el n\'umero de intentos es $a$.
\end{document} 
