\documentclass[letter, 10pt]{article}
\usepackage[latin1]{inputenc}
\usepackage[spanish]{babel}
\usepackage{amsfonts}
\usepackage{amsmath}
\usepackage[dvips]{graphicx}
\usepackage{url}
\usepackage[top=3cm,bottom=3cm,left=3.5cm,right=3.5cm,footskip=1.5cm,headheight=1.5cm,headsep=.5cm,textheight=3cm]{geometry}
\usepackage{listings}
\providecommand{\e}[1]{\ensuremath{\times 10^{#1}}}

\begin{document}
\title{Simulaci\'on \\ \begin{Large}Tarea 3: Metodo de Monte Carlo\end{Large}}
\author{Manuel Almuna Cofr\'e \\
Renato Rivera Mohana.}
\date{\today}
\maketitle
\section{Usando montecarlo evaluar $\int_0^1 \int_0^1 \! e^{x+y} \, \mathrm{d} x\mathrm{d} y $}
Este ejercicio corresponde al caso 4 estudiado, para integrales m\'ultiples, en el que se obtiene $\theta$ mediante $\hat{\theta}= E[g(u_1,u_2,...,u_n)]$
donde $u_1, u_2,...u_n$ sucesion de v.a.i.i.d. $U(0,1)$:
$$\hat{\theta}= \frac{\sum_{i=1}^{k} \! g(u_1^i), g(u_2^i),..., g(u_n^i)}{k} $$

Para realizar el c\'alculo se realiz\'o el siguiente codigo en python:\\
\begin{lstlisting}[language=Python]
from random import random
from math import sqrt

def random_point(lo, hi):
    return [lo[i] + random() * (hi[i]- lo[i]) for i in range(len(lo))]

def integrate(f, lo, hi, N):
    dim = len(lo)
    MCsum = 0L
    MCsumsq = 0L
    for i in range(N):
        point = random_point(lo,hi)
        fx = f(point)
        MCsum += fx
        MCsumsq += fx*fx
    volume = 1L
    MCsum /= N
    MCsumsq /= N
    for i in range(dim):
        volume *= (hi[i] - lo[i])
    error = volume * sqrt(abs(MCsumsq - pow(MCsum,2))/N)
    return [volume * MCsum, error]
\end{lstlisting}
La funci\'on integrate recibe como parametros una funci\'on a integrar, una lista con los limites inferiores y una con los limites superiores de todas las dimensiones,
tambi\'en recibe la cantidad de puntos a iterar N. Al finalizar retorna el valor calculado, junto con el error.\\
En este caso la funci\'on a integrar corresponde a $e^{x+y}$ en el intervalo [0,1] para $x$ e $y$:
\begin{lstlisting}[language=Python]
def function(point):
    x=point[0]
    y=point[1]
    return pow(e,(x+y)*(x+y))

lo = [0,0] # Limites inferiores
hi = [1,1] # Limites superiores
\end{lstlisting}
y evaluamos la integral para distintos N:
\begin{lstlisting}[language=Python]
>>> integrate(function, lo, hi, 10000) # N = 1e4
[4.895196670852154, 0.05879102980391075]

>>> integrate(function, lo, hi, 100000) # N = 1e5
[4.903027602380265, 0.018741319706710415]

>>> integrate(function, lo, hi, 1000000) # N = 1e6
[4.90064772198951, 0.005956517639587255]

>>> integrate(function, lo, hi, 10000000) # N = 1e7
[4.8989669316208, 0.00188453653257943]

>>> integrate(function, lo, hi, 100000000) # N = 1e8
[4.899391481167716, 0.0005960138670439928]
\end{lstlisting}

Finalmente la mayor precisi\'on se alcanza con $N = 1 \e{8}$, obteniendose:
$$\hat{\theta} =\int_0^1 \int_0^1 \! e^{x+y} \, \mathrm{d} x\mathrm{d} y = 4.899391481167716 \pm 0.0005960138670439928$$
\section{Caso 6: Derive un m\'etodo aproximado para calcular $\theta$, v\'ia el m\'etodo Monte Carlo y proponga un algoritmo
$$\theta = \int_{-\infty}^{+\infty} \! g(x) \, \mathrm{d} x $$}
Una forma para calcular este tipo de integrales es dividir la expresi\'on en dos partes:
$$\theta = \int_{-\infty}^{+\infty} \! g(x) \, \mathrm{d} x = \int_{-\infty}^{0} \! g(x) \, \mathrm{d} x + \int_{0}^{+\infty} \! g(x) \, \mathrm{d} x $$
realizamos las siguientes transformaciones de variables para cada parte:\\

$$y= \frac{1}{1-x} \,, \,\,\,\, dy= \frac{dx}{(1-x)^2} = y^{2}dx \,\,\,\,\,  -\infty \leq x \leq 0$$\\
$$z= \frac{1}{1+x} \,, \,\,\,\, dz= \frac{dx}{(x+1)^2} = -z^{2}dx \,\,\,\,\,  0 \leq x \leq \infty$$
entonces,
$$\theta = \int_0^1 \! \frac{g(1 - 1/y)}{y^2} \, \mathrm{d} y \, + \int_{0}^{1} \! \frac{g(1/z - 1)}{z^2} \, \mathrm{d} z $$
\\
Para el calculo de la integral se propone utilizar el algoritmo realizado en la pregunta 1, junto con un complemento que ejecute las transformaciones antes de usar la funci\'on de integraci\'on.\\
Se crea una funci\'on de transformaci\'on para cada tramo:
\begin{lstlisting}[language=Python]
 def gy(point):
    y=point[0]
    x = (1/y) - 1
    return function([x])/(y*y)


def gz(point):
    z=point[0]
    x = 1 - (1/z)
    return function([x])/(z*z)
\end{lstlisting}
gy realiza la transformaci\'on para los valores $-\infty \leq x \leq 0$ y gz para $0\leq x \leq \infty $, ambas luego de realizar la transformaci\'on ejecutan function (funci\'on a integrar) con las nuevas coordenadas. Finalmente el valor de la integral se calcula con la funci\'on:
\begin{lstlisting}[language=Python]
 def iintegrate(N):
    a = integrate(gy, lo, hi, N)
    b = integrate(gz, lo, hi, N)
    return  [a[0]+b[0], a[1]+b[1]]
\end{lstlisting}
retornando el valor con N puntos aleatorios en $([0,1],[0,1])$ junto a su error.
\section{Usando el m\'etodo para resolver el caso 6 determine el valor de $\theta$, con 15 decimales
$$\theta = \int_{-\infty}^{+\infty} \! \frac{1}{1+x^2} \, \mathrm{d} x $$}
Se debe calcular
$$\theta = \int_0^1 \! \frac{g(1 - 1/y)}{y^2} \, \mathrm{d} y \, + \int_{0}^{1} \! \frac{g(1/z - 1)}{z^2} \, \mathrm{d} \,, \,\,\,\, g(x) = \frac{1}{1+x^2}$$
Usando el algoritmo definido en 2 definimos la funci\'on a integrar como:\begin{lstlisting}[language=Python]
def function(point):
    x = point[0]
    return 1/(1+x*x)
\end{lstlisting}
y utilizamos la funci\'on iintegrate:
\begin{lstlisting}[language=Python]
>>> iintegrate(10000) # N = 1e4
[3.1457439213553453, 0.006428512822300107]

>>> iintegrate(100000) # N = 1e5
[3.140125016884107, 0.0020347821049338534]

>>> iintegrate(1000000) # N = 1e6
[3.1420605531452677, 0.0006425576529390718]

>>> iintegrate(10000000) # N = 1e7
[3.141863916995568, 0.00020337431790602427]

>>> iintegrate(100000000) # N = 1e8
[3.141591843708557, 6.430943160832697e-05]

\end{lstlisting}
El valor obtenido corresponde a:
$$\theta = 3.141591843708557 \pm 0.0000643094316083	 \approx \pi$$
\end{document} 
