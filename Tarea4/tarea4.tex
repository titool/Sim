\documentclass[letter, 10pt]{article}
\usepackage[latin1]{inputenc}
\usepackage[spanish]{babel}
\usepackage{amsfonts}
\usepackage{amsmath}
\usepackage[dvips]{graphicx}
\usepackage{url}
\usepackage{algorithm}
\usepackage{algorithmic}
\usepackage[top=3cm,bottom=3cm,left=3.5cm,right=3.5cm,footskip=1.5cm,headheight=1.5cm,headsep=.5cm,textheight=3cm]{geometry}


\begin{document}
\title{Simulaci\'on \\ \begin{Large}Tarea 4: Procesos Estoc\'asticos\end{Large}}
\author{Renato Rivera Mohana.}
\date{\today}
\maketitle

\section{El proceso W={W(T),t }se llama proceso de Wiener si verifica:}
\begin{itemize}
\item W(0)=0 (v.a nula)
\item {W(t)}
\item Si t
\begin{itemize}
\item Determine la funci\'on de medias y de autocovarianzas de W
\item Es W p.e estacionario de 2 oreden?
\end{itemize}
\end{itemize}
\section{Considere el p.e. X, ARMA(1,1) estacionario e invertible dado:}
Simular el proceso ARMA(1,1) con media nula y varianza.
\section{Genere una cadena de Markov}
Con espacio S y matriz de transici\'on P = [] donde p = P . Conocida p = () ,  siendo b j= 1,2,3..N, n\'umero positivos dados.
Muestre que la cadena de Markov resultante es invertible en el tiempo.

\end{document} 
